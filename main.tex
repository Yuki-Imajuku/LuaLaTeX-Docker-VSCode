\documentclass[a4paper,10pt]{ltjsarticle}

\usepackage{amsmath,amssymb}
\usepackage{bm}
\usepackage{booktabs}
\usepackage{caption}
\usepackage{graphicx}
\usepackage{xurl}
\usepackage{tabularx}
\usepackage{hyperref}
\usepackage{tocloft}

\renewcommand{\cftsecfont}{\sffamily}
\renewcommand{\cftsubsecfont}{\sffamily}
\renewcommand{\cftsecpagefont}{\sffamily}
\renewcommand{\cftsubsecpagefont}{\sffamily}
\newcommand*\diff[1]{\mathop{}\!\mathrm{d}#1}

\hypersetup{setpagesize=false, bookmarksnumbered=true, bookmarksopen=true, colorlinks=true, linkcolor=black, citecolor=red, urlcolor=blue}

\title{Lua\LaTeX を Docker と VSCode で使おう!}
\author{にゃ〜ん}
\date{2020/02/02}

\begin{document}
\maketitle

\tableofcontents
\thispagestyle{empty}
\newpage

\setcounter{page}{1}
\section{本文}
ね、簡単でしょ?

\section{数式}
\begin{align*}
\int \diff{\bm{x}}
\end{align*}

\section{特殊文字}
\copyright
\textyen
\textregistered

\section{表}
表\ref{table:table1}を見ると一目瞭然!

\begin{table}[tb]
\caption{Dockerを使えると、嬉しい}\label{table:table1}
\begin{center}
\begin{tabularx}{\linewidth}{@{}XXX@{}} \toprule
なんか & 良さげな & 結果 \\ \midrule
ローカル環境~\cite{local} & 汚れる & めんどくさい \\
Docker~\cite{docker} & コンテナだけ & 便利 \\\bottomrule
\end{tabularx}
\end{center}
\end{table}

\section{その他}
またオレ何かやっちゃいました?\footnote{\url{https://very-very.long.longlong.loooooooooooooooong.url.com/hoge/fuga/nyan?query=key-val-items&query_id=123456789}}

\addcontentsline{toc}{section}{参考文献}
\bibliographystyle{ieeetr}
\bibliography{list.bib}

\end{document}
